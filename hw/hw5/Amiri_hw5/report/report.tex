\documentclass{article}
\usepackage{amsmath}
\usepackage{graphicx}
\usepackage{subfig}

\begin{document}
\title{HW5,1D Stencil with Shared Memory}
\author{Roohollah Amiri}
\maketitle

\section{Introduction}
In this program the effect of using shared memory in a gpu architecture .

\section{Compilation and Results}

Just make a build directory and run cmake from it: mkdir build;cd build;cmake ../;make;
The output of program consists of two tables for part A and B. The check flag indicates if the output vector has correct values or not.
The output of running the code on Redhawk:
PART A:
\begin{table}[h]
\centering\caption{CpuVsGpu}\label{my-label}
\begin{tabular}{lccc}
N         & CPU(mS)    & GPU(global)(mS) & GPU(shared)(mS) \\
100       & 0.000954   & 0.035424        & 0.011584        \\
10000     & 0.061989   & 0.015744        & 0.011904        \\
100000    & 0.617981   & 0.023328        & 0.020992        \\
1000000   & 5.390882   & 0.172192        & 0.114112        \\
10000000  & 60.554028  & 1.664640        & 1.041696        \\
100000000 & 595.370054 & 16.582272       & 9.215136       
\end{tabular}
\end{table}
PART B: Num Elements=100000000

\begin{table}[h]
\centering\caption{Gpu with different number of threads/block}\label{my-label}
\begin{tabular}{lccc}
threads/block  & GPU(global)(mS) & GPU(shared)(mS)    \\
16             &  90.372322      &  64.004959         \\
64             &  40.306847      &  15.348064         \\
256            &  11.405152      &  6.742432          \\
512            &  11.100032      &  7.432704          \\
1024           &  14.619360      &  8.975424 
\end{tabular}
\end{table}

\section{Conclusion} \label{conclusion}

This examples shows the effect of using shared memory and improving the program by using it.

\bibliography{myReferences}
\bibliographystyle{ieeetr}

%\begin{thebibliography}{4}
%
%\bibitem{osher}
%S.Osher, R.Fedkiw, 
%\emph{Level Set Methods and Dynamic Implicit Surfaces}. 
%Springer-Verlag New York, Inc, 2003.
%
%\bibitem{sethian}
%J.A.Sethian, 
%\emph{Level Set Methods and Fast Marching Methods}, 
%Cambridge University Press, 1999.
%
%
%\bibitem{jones}
%M.W. Jones, J.A. Bærentzen, M.Šrámek, 
%\emph{3D Distance Fields: A Survey of Techniques and Applications, Visualization and Computer Graphics}, 
%IEEE Transactions on. 12 Issue: 4 (2006) 581-599.
%
%\end{thebibliography}

\end{document}

